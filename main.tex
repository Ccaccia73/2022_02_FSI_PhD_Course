\documentclass{beamer}

\usetheme{polimi}
% Usage instructions can be found at: https://github.com/elauksap/beamerthemepolimi

\usepackage[utf8]{inputenc}

\title{Mathematical Methods for the FSI Problem}
\subtitle{PhD Course}
\author{Claudio Caccia}
\date{\today}

\begin{document}

\begin{frame}
    \maketitle
\end{frame}

\section{Articles}

\begin{frame}{Selected Articles:}
    
    \begin{itemize}
        \item Leroyer, Alban, et al. \textbf{Efficient FSI codes coupling with possible large added mass effects: applications to rigid and elongated flexible bodies in the maritime field.} COUPLED VI: proceedings of the VI International Conference on Computational Methods for Coupled Problems in Science and Engineering. CIMNE, 2015.
        \item Hagmeyer, Nora, et al. \textbf{Fluid-beam interaction: Capturing the effect of embedded slender bodies on global fluid flow and vice versa.} arXiv preprint arXiv:2104.09844 (2021).
    \end{itemize}
    
\end{frame}

\section{Efficient FSI codes coupling with large AME}



\begin{frame}{Efficient FSI}
\huge
\centering
Efficient FSI codes coupling with possible large added mass effects: \\
applications to rigid and elongated flexible bodies in the maritime field
    
\end{frame}

\subsection{Introduction}

\begin{frame}{Motivation}

\begin{itemize}
    \item FSI in \textit{naval architecture}
    \item Advantages of \textbf{partitioned approach}
    \begin{itemize}
        \item \textbf{ISIS-CFD} for the fluid domain (LHEEA Laboratory Nantes)
        \item \textbf{MBDyn} for the structural domain (Politecnico di Milano)
    \end{itemize}
    \item issues in obtaining a \textit{robust} and \textit{efficient} algorithm due to \textbf{Added Mass Effect} (AME)
    \item Application to \textit{rigid} and \textit{flexible} bodies
\end{itemize}
    
\end{frame}


\subsection{solvers}

\begin{frame}{ISIS-CFD}
    \begin{itemize}
        \item solution of \textit{incompressible} \textbf{RANS}
        \item \textbf{FVM}, with SIMPLE pressure-velocity coupling:
        \begin{itemize}
            \item \textit{velocity} given by momentum equation
            \item \textit{pressure} given by mass conservation (transf. into \textbf{p} equation)
        \end{itemize}
        \item \textbf{ALE} formulation to deal with domain modification (mesh deformation technique not specified, defined \textit{robust})
        \item \textit{temporal discretization} \textbf{BDF2}
        \item internal nonlinearities dealt with \textit{Picard linearisation}
        \item \textbf{VOF} capabilites, surface capturing to describe free surface flows
        
     
    \end{itemize}
\end{frame}

\begin{frame}{MBDyn}
    \begin{itemize}
        \item Solution of complex multibody systems: e.g. nonlinear dynamics, aeroservoelasticity, piezo-electric, hydraulic components, etc...
        \item \textit{Redundant Coordinate Set} (RCS) formulation: \textit{nodes} with 6 dofs
        \item \textit{holonomic} and \textit{nonholonomic} constraints add equations equivalent to \textbf{Lagrange multipliers} which represent reaction force and couples
        \item The complete system is described by a set of first order \textit{Differential Algebraic Equations} (DAE)
        \item a tunable \textit{A-Stable} multistep integration scheme is used to damp numeric oscillations. (\textbf{BDF2} is used)
        \item \textit{interface handling?}
    \end{itemize}
\end{frame}

\subsection{coupling}

\begin{frame}{Coupling description}
on $\Gamma$:
    \begin{itemize}
        \item $\delta_f = \delta_s$  kinematic condition
        \item $\sigma_f + \sigma_f = 0$ dynamic condition
    \end{itemize}


\textbf{Steklov-Poincaré} operator $\mathcal{S}$ maps the values of one BC of the solution of a PDE in a domain to the values of another BC. Here: $\mathcal{S}_d(\delta_d) = \sigma_d \quad \mathcal{S}_d^{-1}(\sigma_d) = \delta_d$
\end{frame}

\subsection{applications}





\section[Fluid-beam interaction]{Fluid-beam interaction: capturing the effect of embedded slender bodies}

\subsection{Introduction}

\begin{frame}{Motivation}
    
\end{frame}

\end{document}
